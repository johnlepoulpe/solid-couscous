\section{Lemmes et remarques intermédiaires}

\subsection{Arbre de hauteur 2, sans modification de hauteur}

Nous nous intéressons ici aux arbres de hauteur 2 dont les feuilles
sont toutes de profondeur 2. Notons $\mathcal{T}_{2}$ l'ensemble de
ces arbres.

\begin{center}
\begin{forest}
  for tree={
    circle,
    draw
  }
  [
    [
      []
      []
      []
    ]
    [
      []
    ]
    [
      []
      []
    ]
  ]
\end{forest}
\end{center}
%Insérer un dessin

Nous imposons plusieurs contraintes à la distance d'édition:
premièrement, nous n'autorisons pas l'ajout ou la suppression de nœuds
internes; de plus, nous imposons aux arbres comparés d'avoir la même
hauteur.

Soit $T \in \mathcal{T}_{2}$. Numérotons
$(u_{i}^{(T)})_{1 \leqslant i \leqslant N}$ les noeuds de profondeur 1
de $T$. Notons $(n_{i})_{1 \leqslant i \leqslant N}$ leur
nombre de fils respectifs. \\
Soit $S = NST(T)$, nous noterons
$(u_{i}^{(S)})_{1 \leqslant i \leqslant N}$ les images des
$(u_{i}^{(T)})_{1 \leqslant i \leqslant N}$ par le mapping. \\
Notons pour $k \in \mathbb{N}$,
$I_{k} = \{ i \in \llbracket 1;N \rrbracket | h(u_{i}^{(S)}) = k\}$
les indices des nœuds de profondeur 1 dont le mapping est de hauteur $k$.\\

\begin{rem}
\label{rem1}
$S$ étant de hauteur 2, dans cette section, $(I_{0}, I_{1})$ forme une
partition de $\llbracket 1;N \rrbracket$.
\end{rem}

\begin{lem}
  \label{lem1}
  Notons $\tilde{n}$ le nombre de fils des sous arbres de hauteur 1
  dans $S$.\\
  Nous avons la formule suivante:
  $$D(T,S) = \sum_{i \in I_{0}} n_{i} + \sum_{i \in I_{1}} |n_{i} - \tilde{n}|$$
  
  \begin{proof}
    Pour obtenir $S$ à partir de $T$, il faut supprimer les fils de
    $u_{i}^{(T)}$ pour tout $i \in I_{0}$, soit $\sum_{i \in I_{0}} n_{i}$
    nœuds au total. De plus pour chaque $i \in I_{1}$,
    \begin{itemize}
    \item si $n_{i} > \tilde{n}$, il faut supprimer
      $n_{i} - \tilde{n}$ fils de $u_{i}^{(T)}$,
    \item si $n_{i} < \tilde{n}$, il faut en ajouter
      $\tilde{n} - n_{i}$, 
    \end{itemize}
    soit un total de $\sum_{i \in I_{1}} |n_{i} - \tilde{n}|$ ajout
    ou suppression. D'où le résultat.
  \end{proof}
\end{lem}

\begin{lem}
  \label{lem2}
  Soit $(I_{0}, I_{1})$ une partition de $\llbracket 1;N \rrbracket$
  fixée (non nécessairement optimale).\\
  $\sum_{i \in I_{0}} n_{i} + \sum_{i \in I_{1}} |n_{i} - \tilde{n}|$
  atteint son minimum en $\tilde{n}$ l'une des médianes des
  $(n_{i})_{i \in I_{1}}$.
\end{lem}

\begin{rem}
  \label{rem2}
  Dans le cadre de notre étude, $\tilde{n}$ doit prendre une valeur
  entière. Aussi, si $I_{1}$ est de cardinal pair et que les deux
  valeurs médianes sont distinctes, nous fixerons $\tilde{n}$ à la
  plus grande des deux valeurs. Ce choix ne modifie en effet pas la
  distance.
\end{rem}

\begin{lem}
  \label{lem3}
  Soit $(I_{0}, I_{1})$ une partition de $\llbracket 1;N \rrbracket$
  fixée (non nécessairement optimale).\\
  Soit $\tilde{n}$ la médiane des $(n_{i})_{i \in I_{1}}$ .
  Soit $S$ l'arbre auto-emboîté défini par $(I_{0}, I_{1})$ et $\overline{n}$.\\
  Supposons qu'il existe $i \in I_{0}$ tel que $n_{i} \geqslant
  \tilde{n}$.
  Soit $S'$ l'arbre auto-emboîté défini par $(I_{0} \setminus \{i\},
  I_{1} \cup \{i\})$ et $\tilde{n}$.\\
  On a l'égalité suivante: $$D(T,S) \geqslant D(T,S')$$
  \begin{proof}
    Calculons la différence:
    \begin{center}
      $
      \begin{array}{rcl}
        D(T,S) - D(T,S') &=& \left( \sum_{k \in I_{0}} n_{k} + \sum_{k \in
                            I_{1}} |n_{k} - \tilde{n}| \right) -\left( \sum_{k \in
                            I_{0}\setminus \{k\}} n_{k} + \sum_{k \in
                            I_{1} \cup \{k\}} |n_{k} - \tilde{n}| \right)\\
                        &=& n_{i} - |n_{i} - \tilde{n}|\\
                        & \geqslant & 0
      \end{array}
      $
    \end{center}
    
  \end{proof}
\end{lem}

\begin{lem}
  \label{lem4}
  Soit $(I_{0}, I_{1})$ une partition de $\llbracket 1;N \rrbracket$
  fixée (non nécessairement optimale).\\
  Soit $\tilde{n}$ la médiane des $(n_{i})_{i \in I_{1}}$.
  Soit $S$ l'arbre auto-emboîté défini par $(I_{0}, I_{1})$ et $\tilde{n}$.\\
  Soit $j = argmin_{k \in I_{1}}n_{k}$.  Supposons qu'il existe
  $i \in I_{0}$ tel que $n_{j} < n_{i} < \tilde{n}$.\\
  Posons $I_{0}' = (I_{0} \setminus \{i\}) \cup \{j\}$ et
  $I_{1}' = (I_{1} \setminus \{j\}) \cup \{i\}$. Soit
  $\tilde{n}'$ la médiane des $(n_{i})_{i \in I_{1}'}$.\\
  Soit $S'$ l'arbre auto-emboîté défini par $(I_{0}', I_{1}')$ et
  $\tilde{n}'$. On a l'inégalité suivante:
  $$D(T,S) \geqslant D(T,S')$$
  \begin{proof}
    Comme $n_{j} < n_{i} < \tilde{n}$, on a $\tilde{n} = \tilde{n}'$.
    On a donc 
    \begin{center}
      $
      \begin{array}{rcl}
        D(T,S) - D(T,S') &=& \left( \sum_{k \in I_{0}} n_{k} + \sum_{k \in
                            I_{1}} |n_{k} - \tilde{n}| \right) -\left( \sum_{k \in
                            I_{0}'} n_{k} + \sum_{k \in
                            I_{1}'} |n_{k} - \tilde{n}| \right)\\
                        &=& n_{i} - n_{j} + |n_{j} - \tilde{n}| -|n_{i} - \tilde{n}|\\
                        & \geqslant & 0
      \end{array}
      $
    \end{center}
    car $n_{j} < n_{i} < \tilde{n}$.
  \end{proof}
\end{lem}

\begin{algorithm}
\label{algo1}
\caption{Compute the NST of a tree of height 2}
\begin{algorithmic} 
\REQUIRE $(n_{i})_{1 \leqslant i \leqslant N}$ in ascending order.
\STATE $M_{min} \leftarrow 0$
\STATE $\tilde{n} \leftarrow n_{\left\lceil \frac{N}{2} \right\rceil}$ 
\STATE $current\_dist \leftarrow \sum_{i=1}^{N} |n_{i} - \overline{n}|$
\STATE $dist_{min} \leftarrow current\_dist$
\FOR{$k \in \llbracket 1;N \rrbracket$}
\STATE $\tilde{n} \leftarrow n_{\left\lceil \frac{N+k}{2} \right\rceil}$ 
\STATE $current\_dist \leftarrow \sum_{i = 1}^{k} n_{i} + \sum_{i=
  k+1}^{N} |n_{i} - \tilde{n}|$
\IF{$current\_dist < dist_{min}$}
\STATE $dist_{min} \leftarrow current\_dist$
\STATE $M_{min} \leftarrow k$
\ENDIF
\ENDFOR
\RETURN $(dist, M_{min}, n_{\left\lceil \frac{N + M_{min}}{2} \right\rceil})$ 
\end{algorithmic}
\end{algorithm}

\begin{thm}
  \label{thm1}
  L'algorithme \ref{algo1} calcule l'arbre auto-emboité le plus proche de $T \in
  \mathcal{T}_{2}$ donné en entrée, ainsi que la distance les
  séparant. De plus, il s'éxécute en temps polynomial en $N$.
  \begin{proof}
    Soit $T \in \mathcal{T}_{2}$. Soit $(I_{0}, I_{1})$
    optimals. D'après les lemmes \ref{lem3} et \ref{lem4}, pour tout
    $i \in I_{0}$, pour tout $j \in I_{1}$, $n_{i} < n_{j}$.\\
    Donc $(I_{0},I_{1})$ sépare les
    $(n_{i})_{i \in \llbracket 1;N \rrbracket}$ en deux parties, l'une
    dont les élements sont supérieurs à $M \in \mathbb{N}$, l'autre
    dans laquelle ils sont inférieurs à $M$. Par conséquent, en
    ordonnant les $(n_{i})_{i \in \llbracket 1;N \rrbracket}$ par
    ordre croissant et en augmentant $M$ progressivement, la partition
    optimale sera atteinte.
  \end{proof}
\end{thm}

\subsection{Arbre de hauteur 2, avec modification de hauteur}

Nous enlevons à présent la contrainte de non-variation de la hauteur
des sous arbres conservés.

\begin{lem}
  \label{lem5}
  Soit $T \in \mathcal{T}_{2}$. Notons $S = NST(T)$. \\
  Tout nœud $u \in S$ de profondeur non nulle (autre que la racine),
  posséde au plus un fils de hauteur non nulle.

  \begin{proof}
    Raisonnons par l'absurde. Supposons qu'il existe un nœud $u$ de
    profondeur non nulle possédant au moins deux fils $v_{1}$ et
    $v_{2}$ de hauteur non nulle. Quitte à inverser $v_{1}$ et
    $v_{2}$, on peut supposer que $h(v_{1}) \geqslant h(v_{2})$.

    Considérons l'arbre $S'$ obtenu depuis $S$ en réduisant à une
    feuille un sous arbre isomorphe à $S[v_{2}]$, dans
    chaque sous-arbres isomorphes à $S[u]$.

    \begin{center}
      \begin{forest}
        for tree={
          circle,
          draw
        }        
        [
          [
            [$u$
              [$v_{1}$
                [
                  []
                ]
              ]
              [$v_{2}$
                []
              ]
              []
            ]
            []
            []
          ]
          [
            [
              [
                []
              ]
            ]
            [
              []
            ]
            []
          ]
          [
            [
              []
            ]
          ]
        ]
      \end{forest}
      \quad
            \begin{forest}
        for tree={
          circle,
          draw
        }        
        [
          [
            [$u$
              [$v_{1}$
                [
                  []
                ]
              ]
              [$v_{2}$]
              []
            ]
            []
            []
          ]
          [
            [
              [
                []
              ]
            ]
            []
            []
          ]
          [
            [
              []
            ]
          ]
        ]
      \end{forest}

    \end{center}
    $S'$ est alors auto-emboîté et son nombre de sommet de profondeur
    supérieure à 2 est strictement inférieur à celui dans $S$. Or tout
    les nœuds de profondeur supérieure à 2 n'appartiennent pas à $T$
    qui est de hauteur 2 (découle de la propriété de non
    ajout/suppression de nœuds internes). 

    Par conséquent, $D(T,S') < D(T,S)$ ce qui est absurde.
  \end{proof}
\end{lem}

\begin{lem}
  \label{lem6}
  Soit $T \in \mathcal{T}_{2}$. Notons $S = NST(T)$. \\
  S'il existe un nœud $u \in S$ de prondeur 1 et de hauteur $k$, avec
  $k \geqslant 2$, alors il existe un nœud $v \in S$ de profondeur 1
  et de hauteur $k-1$.
  
  \begin{proof}
    Raisonnons par l'absurde. Supposons qu'il existe un nœud
    $u_{1} \in S$ de profondeur 1 et de hauteur $k$, avec $k \geqslant 2$, mais
    aucun nœud $v \in S$ de profondeur 1 et de hauteur $k-1$.

    $u_{1}$ étant de hauteur $k \geqslant 2$, $u_{1}$ possède un fils de hauteur non
    nulle. D'après le lemme précédent, ce dernier est unique, nommons
    le $u_{2}$. Deux cas de figures sont alors possibles:
    \begin{enumerate}
    \item $u_{2}$ est de hauteur $k - 1 = 1$. Notons alors $u_{3}$ l'un de
      ses fils.
    \item $u_{2}$ est de hauteur $k - 1 \geqslant 2$. Par le même raisonnement
      que pécédemment, notons $u_{3}$ l'unique fils de $u_{2}$ de hauteur non-nulle. 
    \end{enumerate}

    Considérons $S'$ l'arbre obtenu depuis $S$ en remplacant les
    sous-arbres isomorphes à $S[u_{2}]$ par $S[u_{3}]$. %insérer un dessin
    
    \begin{center}
      \begin{forest}
        for tree={
          circle,
          draw
        }        
        [
          [
            [
              [
                [
                  [
                   []
                  ]
                ]
                []
              ]
              []
              []
            ]
          ]
          [$u_{1}$
            [$u_{2}$
              [$u_{3}$
                [
                  []
                ]
              ]
              []
            ]
            []
            []
          ]
          [
            [
              []
            ]
          ]
          [
            []
          ]
          [
            []
          ]
        ]
      \end{forest}
      \quad
      \begin{forest}
        for tree={
          circle,
          draw
        }        
        [
          [
            [
              [
                [
                 []
                ]
              ]
              []
            ]
          ]
          [$u_{1}$
            [$u_{3}$
              [
                []
              ]
            ]
            []
          ]
          [
            [
              []
            ]
          ]
          [
            []
          ]
          [
            []
          ]
        ]
      \end{forest}
    \end{center}
    $S'$ est alors auto-emboîté car $S$ ne contenait pas de nœud de
    profondeur 1 et de hauteur $k - 1$. 

    De plus le nombre de sommets de $S'$ de profondeur supérieure à 3
    est strictement inférieur à celui dans $S$. Or tout les nœuds de
    profondeur supérieure à 3 n'appartiennent pas à $T$ qui est de
    hauteur 2 (découle de la propriété de non ajout/suppression de
    nœuds internes). De plus, le nombre de nœuds de profondeur
    inférieur à 2 est inchangé car aucun des nœuds de $S$ de profondeur 1
    n'est de hauteur $k - 1$ et donc n'est remplacé par $S[u_{3}]$.

    Par conséquent, $D(T,S') < D(T,S)$ ce qui est absurde.
  
  \end{proof}
\end{lem}

\begin{prop}
  \label{prop1}
  Soit $T \in \mathcal{T}_{2}$. Soit $S = NST(T)$. \\
  En notant $n_{min} = min_{1 \leqslant i \leqslant N} n_{i}$,
  nous avons la formule suivante:
  $$h(S) \leqslant \frac{3}{2} +\sqrt{\frac{9}{4} + 4 \left( \left(\sum_{i \in
          \llbracket 1;N \rrbracket}n_{i} \right) - N n_{min} \right)} $$ 
  c'est-à-dire
  $$h(S) \leqslant \frac{3}{2} +\sqrt{\frac{9}{4} + 4 \left( |T| - N
      (n_{min} + 1) -1 \right)} $$  
  \begin{proof}
    $T$ est de hauteur 2. \\
    Pour atteindre une hauteur $h(S)$, il faut que pour tout
    $k \in \llbracket 2;h(S)-1 \rrbracket$, il existe dans $S$ un nœud
    $u_{i}^{S}$ de profondeur 1 et de hauteur $k$.  Il faut donc
    ajouter au moins $\sum_{k=1}^{h(S) - 2}k =
    \frac{(h(S) -2)(h(S)-1)}{2}$ à $T$ pour obtenir $S$.\\
    Considérons l'arbre $S'$ obtenu depuis $T$ en supprimant des
    feuilles jusqu'à ce que pour tout $i$, $n_{i}^{S} = n_{min}$. $S'$ est
    un arbre auto-emboîté dont la distance à $T$ est
    $$D(T,S') = \left(\sum_{i \in \llbracket 1;N \rrbracket}n_{i} \right) - N n_{min} = |T| - N (n_{min} + 1) - 1$$
    On a donc
    $D(T,S') \geqslant D(T,S) \geqslant \frac{(h(S)
      -2)(h(S)-1)}{2}$.
    En résolvant cette équation du second degré en $h(S)$, nous obtenons
    $$h(S) \leqslant \frac{3}{2} +\sqrt{\frac{9}{4} + 4 D(T,S')}$$
  \end{proof}
\end{prop}

\subsection{Work in progress}
\begin{lem}
  \label{lem7}
  Soit $T \in \mathcal{T}_{2}$. Soit $S = NST(T)$. Notons pour tout
  $k \in \mathbb{N}$, $I_{k}$ l'ensemble des indices
  $i \in \llbracket 1;N \rrbracket$ pour lesquels $u_{i}^{(S)}$ est de
  hauteur $k$. \\
  Notons $H = h(S)$. Notons pour tout
  $k \in \llbracket 1;H \rrbracket$, $\tilde{n}^{k}$ le
  nombre de fils des nœuds de prondeur 1 et de hauteur $k$ dans $S$.\\
  On a la formule suivante :
  $$ D(S,T) = \sum_{i \in I_{0}} n_{i} + \sum_{k = 1}^{H} \left(
    \sum_{i \in I_{k}} |n_{i} - \tilde{n}^{k}| + |I_{k}|\sum_{l =
      1}^{k-1}(\tilde{n}^{l}) \right)$$
  \begin{proof}
    Pour obtenir $S$ à partir de $T$, il faut supprimer les fils de
    $u_{i}^{(T)}$ pour tout $i \in I_{0}$, soit $\sum_{i \in I_{0}} n_{i}$
    nœuds au total. De plus pour chaque $i \in I_{k}$,
    \begin{itemize}
    \item si $n_{i} > \tilde{n}^{k}$, il faut supprimer
      $n_{i} - \tilde{n}^{k}$ fils de $u_{i}^{(T)}$,
    \item si $n_{i} < \tilde{n}^{k}$, il faut en ajouter
      $\tilde{n}^{k} - n_{i}$. 
    \end{itemize}
    L'un des fils de $u_{i}^{S}$ est de hauteur $k-1$, il faut donc
    ajouter de plus $\sum_{l = 1}^{k-1}(\tilde{n}^{l})$ nœuds sous
    celui-ci. En effet, d'après le lemme \ref{lem5}, les sous-arbres
    de hauteur $k \geqslant 1$ de $S$ contiennent
    $\tilde{n}^{k-1} + \left| S^{(k-1)} \right|$ nœuds, où
    $ \left| S^{(k-1)} \right|$ est le nombre de nœuds des sous arbres
    de $S$
    de hauteur $k-1$. On montre alors aisément par récurrence que
    $\left| S^{(k-1)} \right| = 1 + \sum_{l =
      1}^{k-1}(\tilde{n}^{l})$.\\
    Soit un total de
    $\sum_{i \in I_{1}} |n_{i} - \tilde{n}| + |I_{k}|\sum_{l =
      1}^{k-1}(\tilde{n}^{l})$ ajout ou suppression. D'où le résultat.
 
  \end{proof}
\end{lem}

\begin{rem}
  \label{rem3}
   $$ D(S,T) = \sum_{i \in I_{0}} n_{i} + \sum_{k = 1}^{H} \left(
    \sum_{i \in I_{k}} |n_{i} - \tilde{n}^{k}| + \tilde{n}^{k}\sum_{l =
      k+1}^{H}(|I_{l}|) \right)$$
\end{rem}

\begin{lem}
  \label{lem8}
  Soit $T \in \mathcal{T}_{2}$. Soit $S = NST(T)$.  Soient
  $(I_{k})_{0 \leqslant k \leqslant h(S)}$ et
  $(\tilde{n}^{k})_{1 \leqslant k \leqslant h(S)}$ les suites
  associées.\\
  Pour tout $k \in \llbracket 1;H-1 \rrbracket$, $$|I_{k}| > \sum_{l=
      k+1}^{H}|I_{l}|$$
    \begin{proof}
      Supposons le contraire, soit $k \in \llbracket 1;H-1 \rrbracket$
      tel que $|I_{k}| \leqslant \sum_{l= k+1}^{H}|I_{l}|$.Posons
    $(J_{l})_{0 \leqslant \leqslant h(S)-1}$ et
    $(\tilde{m}^{\ l})_{1 \leqslant \leqslant h(S)-1}$ où
    \begin{align*}
      J_{l} =
      \begin{cases}
        I_{k} \cup I_{0} &\mbox{si } l=0\\       
        I_{l} &\mbox{si } 1 \leqslant l \leqslant k-1\\
        I_{l+1} &\mbox{sinon.}
      \end{cases}
                  \text{ et } \tilde{m}^{\ l} =
                  \begin{cases}
                    \tilde{n}^{l} &\mbox{si } 1 \leqslant l \leqslant k-1\\
                    \tilde{n}^{l+1} &\mbox{sinon.}
                  \end{cases}
    \end{align*}
    Notons $S'$ l'arbre auto-emboîté associé aux
    $(J_{l})_{0 \leqslant \leqslant h(S)-1}$ et
    $(\tilde{m}^{l})_{0 \leqslant \leqslant h(S)-1}$.
    \begin{center}
      $$
      \begin{array}{rcl}
        D(T,S) - D(T,S') &=& \left( \sum_{i \in I_{k}}
                             |n_{i}-\tilde{n}^{k}|+ |I_{k}|
                             \sum_{l=1}^{k-1} \tilde{n}^{l} + |\tilde{n}^{k}|
                             \sum_{l=k+1}^{h(S)} |I_{l}| \right) -
                             \sum_{i \in I_{k}} n_{i}  \\

                         &=& \sum_{i \in I_{k}} \left(|n_{i} -
                             \tilde{n}^{k}| - n_{i} +\left( \sum_{l=1}^{k-1}
                             \tilde{n}^{l} \right)\right) +
                             \tilde{n}^{k}\sum_{l=k+1}^{H} |I_{l}| \\
                         &\geqslant& \sum_{i \in I_{k}} \left(|n_{i} -
                                     \tilde{n}^{k}| + (\tilde{n}^{k}
                                     -n_{i}) + \left( \sum_{l=1}^{k-1}
                                     \tilde{n}^{l} \right) \right) \\
                         &>& \sum_{i \in I_{k}} \left(|n_{i} -
                             \tilde{n}^{k}| + (\tilde{n}^{k}
                             -n_{i}) \right)\\
                         &>& 0
      \end{array}
      $$
    \end{center}
    Donc $D(T,S) > D(T,S')$ ce qui est absurde car $S'$ est auto-emboîté.
  \end{proof}
\end{lem}

\begin{cor}
  \label{cor1}
  On a pour tout $k \in \llbracket 1;H \rrbracket$, $|I_{k}| \geqslant 2^{H-k}$.
  De plus, en notant $N$ le nombre de nœuds de $T$ de profondeur 1, on a 
  $$h(S) \leqslant \left\lfloor \log_{2}(N+1) \right\rfloor$$
  \begin{proof}
    En notant $H = h(S)$, on a
    \begin{center}
      $$
      \begin{array}{rcl}
        |I_{k}| &\geqslant& \sum_{l=k+1}^{H} |I_{l}| + 1\\
                &\geqslant& |I_{k+1}| + \sum_{l=k+2}^{H} |I_{l}| + 1 \\
                &\geqslant& 2(\sum_{l=k+2}^{H} |I_{l}| + 1) \\
                &\geqslant& 2^{H-k-1}(I_{H} +1)\\
                &\geqslant& 2^{H-k}\\
      \end{array}
      $$
    \end{center}
    Donc $N
    \geqslant \sum_{k=1}^{H} |I_{k}| \geqslant \sum_{k=1}^{H} 2^{H-k}
    \geqslant 2^{H} -1$. \\
    Donc $H \leqslant \left\lfloor \log_{2}(N+1) \right\rfloor$
      \end{proof}
\end{cor}

\begin{lem}
  \label{lem9}
  Soit $T \in \mathcal{T}_{2}$. Soit $S = NST(T)$.  Soient
  $(I_{k})_{0 \leqslant k \leqslant h(S)}$ et
  $(\tilde{n}^{k})_{1 \leqslant k \leqslant h(S)}$ les suites
  associées.\\
  Pour tout $k \in \llbracket 2;H \rrbracket$, $$\tilde{n}^{k} > \sum_{l=1}^{k-1}\tilde{n}^{k}$$
    \begin{proof}
      Supposons le contraire, soit $k \in \llbracket 2;H \rrbracket$
      tel que
      $\tilde{n}^{k} \leqslant \sum_{l= 1}^{k-1} \tilde{n}^{l}$.Posons
      $(J_{l})_{0 \leqslant \leqslant h(S)-1}$ et
      $(\tilde{m}^{\ l})_{1 \leqslant \leqslant h(S)-1}$ où
    \begin{align*}
      J_{l} =
      \begin{cases}
        I_{k} \cup I_{0} &\mbox{si } l=0\\       
        I_{l} &\mbox{si } 1 \leqslant l \leqslant k-1\\
        I_{l+1} &\mbox{sinon.}
      \end{cases}
                  \text{ et } \tilde{m}^{\ l} =
                  \begin{cases}
                    \tilde{n}^{l} &\mbox{si } 1 \leqslant l \leqslant k-1\\
                    \tilde{n}^{l+1} &\mbox{sinon.}
                  \end{cases}
    \end{align*}
    Notons $S'$ l'arbre auto-emboîté associé aux
    $(J_{l})_{0 \leqslant \leqslant h(S)-1}$ et
    $(\tilde{m}^{l})_{0 \leqslant \leqslant h(S)-1}$.
    \begin{center}
      $$
      \begin{array}{rcl}
        D(T,S) - D(T,S') &=& \left( \sum_{i \in I_{k}}
                             |n_{i}-\tilde{n}^{k}|+ |I_{k}|
                             \sum_{l=1}^{k-1} \tilde{n}^{l} + |\tilde{n}^{k}|
                             \sum_{l=k+1}^{h(S)} |I_{l}| \right) -
                             \sum_{i \in I_{k}} n_{i}  \\

                         &=& \sum_{i \in I_{k}} \left(|n_{i} -
                             \tilde{n}^{k}| +\left( \sum_{l=1}^{k-1}
                             \tilde{n}^{l} \right) - n_{i} \right) +
                             \tilde{n}^{k}\sum_{l=k+1}^{H} |I_{l}| \\
                         &>& \sum_{i \in I_{k}} \left(|n_{i} -
                             \tilde{n}^{k}| +\left( \sum_{l=1}^{k-1}
                             \tilde{n}^{l} \right) - n_{i}\right)\\
                         &>& \sum_{i \in I_{k}} \left(|n_{i} -
                             \tilde{n}^{k}| + (\tilde{n}^{k} - n_{i})\right)\\                         
                         &>& 0
      \end{array}
      $$
    \end{center}
    Donc $D(T,S) > D(T,S')$ ce qui est absurde car $S'$ est auto-emboîté.
  \end{proof}
\end{lem}

\begin{cor}
  \label{cor2}
  On a pour tout $k
  \in \llbracket 1;H \rrbracket$, $\tilde{n}^{k} \geqslant
  2^{k}$. \\
  De plus, en notant $n_{max}
  = max_{i \in \llbracket 1,;N \rrbracket} n_{i}$, on a $h(S)
  \leqslant \left\lfloor \log_{2}(n_{max}+1) \right\rfloor$
  \begin{proof}
    En notant $H = h(S)$, on a 
    \begin{center}
      $$
      \begin{array}{rcl}
        \tilde{n}^{k} &\geqslant& \sum_{l=1}^{k-1} \tilde{n}^{l} + 1\\
                      &\geqslant& \tilde{n}^{k-1} + \sum_{l=1}^{k-2}
                                  \tilde{n}^{l} + 1 \\
                      &\geqslant& 2(\sum_{l=1}^{k-2} \tilde{n}^{l} + 1) \\
                      &\geqslant& 2^{k-1}(\tilde{n}^{1} +1)\\
                      &\geqslant& 2^{k}\\
      \end{array}
      $$
    \end{center}
    Donc $n_{max} \geqslant \tilde{n}^{H}
    \geqslant \sum_{k=1}^{H-1} \tilde{n}^{l} \geqslant \sum_{k=1}^{H-1} 2^{k}
    \geqslant 2^{H} -1$. \\
    Donc $H \leqslant \left\lfloor \log_{2}(n_{max}+1) \right\rfloor$
      \end{proof}
\end{cor}

\begin{lem}
  \label{lem10}
  Soit $T \in \mathcal{T}_{2}$. Soit $S = NST(T)$.  Soient
  $(I_{k})_{0 \leqslant k \leqslant h(S)}$ et
  $(\tilde{n}^{k})_{1 \leqslant k \leqslant h(S)}$ les suites
  associées.\\
  Pour tout $k \in \llbracket 1;H \rrbracket$, $\tilde{n}^{k}$
  minimise
  $\sum_{i \in I_{k}} |n_{i} - \tilde{n}^{k}| +
  \tilde{n}^{k}\sum_{l=k+1}^{h(S)}|I_{l}|$.\\ 
  $\tilde{n}^{k}$ est donc la $\left\lfloor \frac{|I_{k}|
      - \sum_{l=k+1}^{h(S)}|I_{l}|}{2} \right\rfloor$ ème plus petite
  valeur de $\{n_{i} | i \in I_{k} \}$.
\end{lem}

\begin{lem}
  \label{lem11}
  Soit $T \in \mathcal{T}_{2}$. Soit $S = NST(T)$.  Soient
  $(I_{k})_{0 \leqslant k \leqslant h(S)}$ et
  $(\tilde{n}^{k})_{1 \leqslant k \leqslant h(S)}$ les suites
  associées.\\
  Pour tout $i \in \llbracket 1;N \rrbracket$, si $\tilde{n}^{k}
  \leqslant n_{i} \leqslant \tilde{n}^{k+1}$ alors $i \in I_{k} \cup
  I_{k+1}$
  \begin{proof}
    D'après le lemme \ref{lem10}
    $(\tilde{n}^{k})_{1 \leqslant k \leqslant h(S)}$ est
    croissante. Par conséquence, $i \in \bigcup_{l \leqslant k+1}
    I_{l}$.\\
    Supposons que $i \in I_{h}$ avec $h \leqslant k-1$.
    
    Notons $S'$ l'arbre auto-emboîté associé aux
    $(J_{l})_{0 \leqslant \leqslant h(S)-1}$ et
    $(\tilde{m}^{l})_{0 \leqslant \leqslant h(S)-1}$.
    \begin{center}
      $$
      \begin{array}{rcl}
        D(T,S) - D(T,S') &=& |n_{i}-\tilde{n}^{h}| -
                             |n_{i}-\tilde{n}^{k}| + \sum_{l=1}^{h-1}
                             \tilde{n}^{l} -\sum_{l=1}^{k-1} \tilde{n}^{l}\\
                         &=& n_{i} - \tilde{n}^{h} -
                             (n_{i}-\tilde{n}^{k}) - \sum_{l=h}^{k-1}
                             \tilde{n}^{l} \\
                         &=& \tilde{n}^{k} - \tilde{n}^{h} -
                             \sum_{l=h}^{k-1} \tilde{n}^{l} \\
                         &>& \sum_{l=1}^{k-1} \tilde{n}^{l} -
                             \tilde{n}^{h} - \sum_{l=h}^{k-1} \tilde{n}^{l} \\
                         &>& \sum_{l=1}^{h-1} \tilde{n}^{l} -
                             \tilde{n}^{h}\\
                         &>&0
      \end{array}
      $$
    \end{center}
    Les inégalités ci-dessus découle du lemme \ref{lem9}. 
    On a donc $D(T,S) > D(T,S')$ ce qui est absurde car $S'$ est auto-emboîté.
  \end{proof}
\end{lem}

\begin{lem}
  \label{lem12}
  Soit $T \in \mathcal{T}_{2}$. Soit $S = NST(T)$.  Soient
  $(I_{k})_{0 \leqslant k \leqslant h(S)}$ et
  $(\tilde{n}^{k})_{1 \leqslant k \leqslant h(S)}$ les suites
  associées.\\
  Pour tout $i \in \llbracket 1;N \rrbracket$, si $\tilde{n}^{k}
  \leqslant n_{i} \leqslant \sum_{l=1}^{k+1}\tilde{n}^{l}$ alors $i \in I_{k}$
  \begin{proof}
    D'après le lemme \ref{lem11}, $i \in I_{k} cup I_{k+1}$.\\
    Supposons que $i \in I_{k+1}$.
    
    Notons $S'$ l'arbre auto-emboîté associé aux
    $(J_{l})_{0 \leqslant \leqslant h(S)-1}$ et
    $(\tilde{m}^{l})_{0 \leqslant \leqslant h(S)-1}$.
    \begin{center}
      $$
      \begin{array}{rcl}
        D(T,S) - D(T,S') &=& |n_{i}-\tilde{n}^{h}| -
                             |n_{i}-\tilde{n}^{k}| + \sum_{l=1}^{h-1}
                             \tilde{n}^{l} -\sum_{l=1}^{k-1} \tilde{n}^{l}\\
                         &=& n_{i} - \tilde{n}^{h} -
                             (n_{i}-\tilde{n}^{k}) - \sum_{l=h}^{k-1}
                             \tilde{n}^{l} \\
                         &=& \tilde{n}^{k} - \tilde{n}^{h} -
                             \sum_{l=h}^{k-1} \tilde{n}^{l} \\
                         &>& \sum_{l=1}^{k-1} \tilde{n}^{l} -
                             \tilde{n}^{h} - \sum_{l=h}^{k-1} \tilde{n}^{l} \\
                         &>& \sum_{l=1}^{h-1} \tilde{n}^{l} -
                             \tilde{n}^{h}\\
                         &>&0
      \end{array}
      $$
    \end{center}
    Les inégalités ci-dessus découle du lemme \ref{lem9}. 
    On a donc $D(T,S) > D(T,S')$ ce qui est absurde car $S'$ est auto-emboîté.
  \end{proof}
\end{lem}


%%%%%%%%%%%%%%%%%%%%%%%%%%%%%%%%%%%%%%%%%%%%%%%%%%%%%%%%%%%%%%%%%%%%%%%%%%%%
\noindent\hrulefill
