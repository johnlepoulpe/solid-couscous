\section*{Introduction}
\addcontentsline{toc}{section}{\protect\numberline{}Introduction}%

The MOSAIC team develops mathematical models and tools to study the
morphogenesis and growth of animals and plants. In order to do that,
they design mathematical models and data structures able to describe
efficiently and accurately the shapes and properties of the plants or
animals.

For example, it is natural to represent the branching structure of a
plant by a tree. In those trees representing plants, we can observe
that some ramifications patterns tend to repeat themselves in many
places. The MOSAIC team has thus designed a theoretical model of the
structure of a plant based on this idea of repeated patterns:
self-nested trees, trees in which all the subtrees of equal height are
isomorphic.

Some biological or physical quantities (such as the flow of sap in the
plant for example) can be computed on these trees. Depending on the
species, the size of the trees representing the plants can be
tremendous and lead to unreasonable computing time. They also
developed a compression algorithm for trees, based on the same idea of
patterns within the tree, by eliminating this redundancy. Moreover, by
eliminating this redundancy, the different patterns in the tree and
their organization are highlighted. For this reason, this compression
helps to understand the structure of the plant and its developement.

Those compression algorithms return DAGs (Directed Acyclic Graph) and
it is possible to show that the tree having the best compression
factor are the self-nested ones. Therefore, it is beneficial to
approximate trees by self-nested ones, in order to have an approximate
yet compact representation of them. There are several ways to do this
and the MOSAIC team has developed different algorithms doing
this. However, the approximation given by those algorithms are not the
best (either because these algorithms are heuristics or because some
additional conditions are imposed on the self nested trees returned by
the algorithm). In fact no efficient algorithm has been found to
compute the nearest self-nested tree to a tree.

Therefore, proving that there exists no polynomial algorithm solving
this problem would justify the use of the non optimal algorithms
previously developed by the team. More precisely, the goal of this
internship is to prove that this problem is NP-complete. NP-complete
problems are the hardest problems of the NP class and under the
assumption that the P and NP complexity classes are different, this
would prove there exists no polynomial algorithm able to find the
nearest self-nested tree to a tree in general.

The state of the art, along with definitions and notations will be
presented and introduced in a first section. We will then give further
details on the NP-complete theory and present the results found during
the intership. Finally, we will explain how this results consist in an
improvement of the state of the art and the consequences and
applications.
