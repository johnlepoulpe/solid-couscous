\section{Contribution}

We assume that the reader is familiar with the complexity theory. In
this article, we will denote the set of instances by $\Omega$ and the
set of positive instances by $S$. See the appendix for further
reminder on the compexity theory.

\subsection{Proof of the belonging to the NP class}

We aim to prove that the NST decision problem is NP-complete. It
clearly belongs to the NP class. Indeed, for all tree $T = (V,E)$, for
all integer $k$, given a potential solution $S$, it is possible to
ascertain whether or not $S$ is self-nested, and to compute the
distance between $S$ and $T$ also in polynomial time.

\subsection{Intuition guiding the researches}
To help us build the reduction, a few conceptual remarks can be made,
in order to guide the research. First of all, the reduction has no
obligation to be surjective, as a matter of fact, it is quite often
not surjective. The image of the reduction has to be small enough to know the
general form of the solution of each instance in order to be able to
prove the equivalence
$\omega_{A} \in S_{A} \Leftrightarrow \omega_{b} = f(\omega_{a}) \in S_{B}$ (where
$A$ is the initial NP-complete problem and $B$ is the problem we
consider). Yet, if the image of $\Omega_{A}$ by the reduction has to
be big enough so that we are enable to find a polynomial algorithm
solving the problem on the set of images of the reduction. Indeed,
finding such an algorithm and a reduction would mean we would have
solved the open question of the equality of the P and NP classes,
which is unreasonable.

Furthermore, we can observe that generally, the construction of the
image instance seems to be combination of two different things: a part
that reflects only the number of variables in the instance of the
first problem and a part reflecting the links and constraints that
they have with one another.  For example, in the proof of the
reduction from 3-SAT to INDEP-SET \cite{polytech}, the first part
consists in the triangles corresponding to each clause of the initial
instance, and the second to the edges linking the nodes such as one is
labeled with the negation of the other. In the proof of the reduction
from VERTEX-COVER to HAMILTONIAN-CYCLE \cite{polytech}, the first part
consists in the 12 nodes widgets corresponding to each edge of the
initial graph, and the second part consists in the way they are linked
to one another to form the image graph.

We can also observe that in the first part, a piece of the image
instance corresponds to each variable of the initial
instance. Moreover, this piece of the image instance is very
restrictive and can be used in a solution only in a few (2 or 3 most
of the time) different ways. For example in the proof of the reduction
from VERTEX-COVER to HAMILTONIAN-CYCLE \cite{polytech}, the widgets
can either be visited in one pass or in two.

Furthermore the way these pieces are used is completely independant
until the second part corresponding to the constraints between the
variables is added.

\subsection{Trees of height 2 without variation of the height}
The edit distance with constraint of K.Zhang being slightly difficult
to manipulate, we will use a far more restrictive variant in this
section: only leaves (and not internal nodes) can be added or
deleted. More precisely, if $u$ is an ancestor of $v$ in $T$ and
$v \in V'_{1}$ then $u \in V'_{1}$.

In this section we present a polynomial algorithm computing the
nearest self-nested tree of height 2 to a tree of height 2.

\subsection{Trees of height 2 with variation of the height}
In this section, we present a polynomial algorithm computing the nearest
self-nested tree to a tree of height 2.

\subsection{Family of bricks}
In this section, we present a family of trees that could be used as
the first part of the reduction to represent the variables of the
initial instance.